\documentclass[twoside]{article}

\usepackage{lipsum} % Package to generate dummy text throughout this template

\usepackage[sc]{mathpazo} % Use the Palatino font
\usepackage[T1]{fontenc} % Use 8-bit encoding that has 256 glyphs
\linespread{1.05} % Line spacing - Palatino needs more space between lines
\usepackage{microtype} % Slightly tweak font spacing for aesthetics

\usepackage[hmarginratio=1:1,top=32mm,columnsep=20pt]{geometry} % Document margins
\usepackage{multicol} % Used for the two-column layout of the document
\usepackage[hang, small,labelfont=bf,up,textfont=it,up]{caption} % Custom captions under/above floats in tables or figures
\usepackage{booktabs} % Horizontal rules in tables
\usepackage{float} % Required for tables and figures in the multi-column environment - they need to be placed in specific locations with the [H] (e.g. \begin{table}[H])
\usepackage{hyperref} % For hyperlinks in the PDF

\usepackage{lettrine} % The lettrine is the first enlarged letter at the beginning of the text
\usepackage{paralist} % Used for the compactitem environment which makes bullet points with less space between them

\usepackage{abstract} % Allows abstract customization
\renewcommand{\abstractnamefont}{\normalfont\bfseries} % Set the "Abstract" text to bold
\renewcommand{\abstracttextfont}{\normalfont\small\itshape} % Set the abstract itself to small italic text

\usepackage{titlesec} % Allows customization of titles
\renewcommand\thesection{\Roman{section}} % Roman numerals for the sections
\renewcommand\thesubsection{\Roman{subsection}} % Roman numerals for subsections
\titleformat{\section}[block]{\large\scshape\centering}{\thesection.}{1em}{} % Change the look of the section titles
\titleformat{\subsection}[block]{\large}{\thesubsection.}{1em}{} % Change the look of the section titles

\usepackage{fancyhdr} % Headers and footers
\pagestyle{fancy} % All pages have headers and footers
\fancyhead{} % Blank out the default header
\fancyfoot{} % Blank out the default footer
\fancyfoot[RO,LE]{\thepage} % Custom footer text

%----------------------------------------------------------------------------------------
%	TITLE SECTION
%----------------------------------------------------------------------------------------

\title{\vspace{-15mm}\fontsize{18pt}{10pt}\selectfont\textbf{An Evaluation of a C0 to Asm.js Compiler}} % Article title

\author{
\large
\textsc{Laxman Dhulipala, Peter Xiao}\\[2mm] % Your name
\normalsize Carnegie Mellon University \\ % Your institution
\normalsize \href{mailto:ldhulipa@andrew.cmu.edu}{ldhulipa@andrew.cmu.edu}, \href{mailto:phx@andrew.cmu.edu}{phx@andrew.cmu.edu} % Your email address
\vspace{-5mm}
}
\date{}

%----------------------------------------------------------------------------------------

\begin{document}

\maketitle % Insert title

\thispagestyle{fancy} % All pages have headers and footers

%----------------------------------------------------------------------------------------
%	ABSTRACT
%----------------------------------------------------------------------------------------

\begin{abstract}

\noindent We evaluate an experimental compiler that translates C0 code into Asm.js, a strict
and highly optimizable subset of Javascript. We describe the results both holistically and empirically,
profiling performance using the Spidermonkey and V8 JavaScript engines. Importantly, in the
translation, we preserve all dynamic semantics guaranteed by C0 and guarantee deterministic execution.

\end{abstract}

%----------------------------------------------------------------------------------------
%	ARTICLE CONTENTS
%----------------------------------------------------------------------------------------

\begin{multicols}{2} % Two-column layout throughout the main article text

\section{Introduction}

%\lettrine[nindent=0em,lines=3]{L} orem ipsum dolor sit amet, consectetur adipiscing elit.
%\lipsum[2-3] % Dummy text

Asm.js is a recent effort by Mozilla to develop a highly optimizable and performant
subset of JavaScript. The motivation behind the effort stems from a series of projects which
compile C and C++ to JavaScript, most notably the Emscripten compiler. With the availability of such
compilers, the only missing features for browsers was an engine which could run the
translated native code at near-native speeds. Asm.js is Mozilla's solution to this problem,
and provides a directive triggered mechanism by which the browser can defer control to a
self-contained, non-garbage collected module that runs translated C and C++ at near-native speeds. \\
\\
As we will shortly see, Asm.js is not a language that is fun or easy for humans to write in -
its use is primarily as a compiler target. Furthermore, Asm.js is still in a very `alpha' stage. 
The spec is not formal in the slightest, and important details about what types of expressions can
be validated are left missing, and can only be discovered by trial and error. A number of these issues
can be found in the final sections of our report. The current mechanism for compiling C, and C++ to
JavaScript is as follows:

\begin{compactitem}
  \item C and C++ is compiled to the LLVM IR
  \item The LLVM IR is handed to Emscripten, a LLVM to JavaScript compiler
  \item Emscripten compiles the IR to a portable asm.js module
\end{compactitem}

\section{Motivation}

There are many reasons for wanting to write a C0 to Asm.js compiler. For one,
Asm.js is a language that is being actively worked on by researchers at Mozilla.
The goals of this project are to generate incredibly optimized ahead-of-time 
compiled code that the Javascript interpreter can then run at native speeds.

From our perspective, our project was a useful experiment because it served 
as a proof of concept that C0 as a language is strict enough, and well-behaved
enough to embed into a different language, with different runtime semantics, 
and still emulate the dynamic semantics of the original language. This shows
both that C0 is strict enough, and Asm.js is flexible enough to permit the 
embedding. 

Furthermore, having essentially Javascript that reproduces the behavior of a 
C0 program is invaluable. As the web is arguably the biggest code deployment 
platform, and our implementation is compliant with the Ecmascript standards, 
this means that we can run our C0 code \emph{anywhere}. From the perspective
of a teacher, this could easily lead to an in-browser editor where students 
could write C0, compile the code, and run it, all without ever leaving the 
browser.

%------------------------------------------------

\section{Project Specification}

At a high level, this project will re-target our L5 compiler (the L4 language,
with optimizations) to JavaScript that conforms to Asm.js specifications. In other
words, all of the following features C0 will be supported: loops, conditionals,
functions, memory, pointers, arrays, structs, external library functions, and
exceptions.

  We built our compiler on top of the L5 frontend. This ensures that all 
  frontend optimizations implemented in L5 carry over to this lab. Invoking
  the compiler with optimization flags, and not passing the \texttt{--asmjs}
  directive will still compile C0 code to assembly. We guaranteed that all
  normal functionality is maintained, as in L5. 

  Memory is supported as in L5 with no garbage collection. Exceptions are
  still thrown as they would in previous labs, but execution of the program
  will not halt.

  All of the external libraries in C0 are reimplemented in JS, so that the functions
  are still callable. We go into further detail regarding the external float library
  in a subsequent section. 

As always, correctness is paramount. Our number one goal is to create a correctly
functioning compiler. Finally, we describe how we assess correctness in section V. 

%------------------------------------------------

\section{Implementation}

  \subsection{Integers}
    All variables in asm.js must be explicitly annotated with type information.
    This is so that their type will be statically enforced.

    Integers are annotated with an \texttt{| 0} appended to the end of any
    assignment or return. Operations on integers, such as \texttt{+} force
    the \texttt{Int} from an \texttt{Int} into \texttt{Intish}, a subtype of \texttt{Int}. 
    This requires an explicit coercion back to \texttt{Int}.

    There are also a number of special operations on ints, which must be
    wrapped in function calls. Specifically, we have function calls for \texttt{Mul},
    \texttt{Div}, \texttt{Mod}, \texttt{ShiftL} and \texttt{ShiftR}. 
    
    Mul requires a 
    wrapper because of the a quirk in the way asm.js and firefox handle C-like 32-bit integer 
    multiplication of two operands. Instead of allowing `*' to do this, they offload the work
    onto a function the asm.js module expects from stdlib, namely \texttt{stdlib.imul}. In 
    our implementation, we check whether or not the global environment contains an imul function,
    and replace imul with a polyfill if it is missing. 

    Div and Mod require wrappers because we must check to see whether the denominators are 0,
    and set the numerical exception flag to \texttt{1} if this is the case. Furthermore, we have
    a special case for \texttt{MIN\_INT} divided by, or mod by, \texttt{-1}, as this should raise
    a numerical exception when using C-like 32-bit ints. 

    Left-Shift and Right-Shift also require wrapper functions, as their safe-mode operation
    is contingent on the shift-value being strictly between $[0,32)$. The functions check
    to see whether this is true, and set the numerical exception flag if the shift-operand is
    not in this range. 
  \subsection{Functions}
    Functions in Asm.js are very similar to their C0 counterparts, but with a
    slightly different syntax. Function calls require explicit coercions. Furthermore,
    function arguments must have their types be immediately declared. For example, for
    a function \texttt{mult} which takes two arguments, and returns the result of applying
    \texttt{stdlib.imul} to them, we have the following code: 
\begin{verbatim}
function mult(fst, snd) {
  fst = fst | 0;
  snd = snd | 0;
  return stdlib.imul((fst | 0), 
                      (snd | 0)) | 0;
}
\end{verbatim}

  \subsection{Loops and Conditionals}
    These are very similar to their C0 counterparts.

  \subsection{Booleans and Boolean Operands}
    These are simulated by use of 0 and 1. Asm.js has the strange quirk of disallowing 
    \texttt{\&\&} and \texttt{||} operators. We however, noticed that performance in browsers
    and engines that do not support Asm.js suffered quite drastically from not using \texttt{\&\&}
    and selectively used \texttt{\&\&} - this is however, an easy fix (four poly-fill functions we
    hand-wrote) if we want to produce code that is fully compliant to the Asm.js spec. 

  \subsection{Memory}
    Asm.js provides a large binary heap, in the form of a typed-array (\texttt{ArrayBuffer}). 
    This can then be cast into any array-view the user desires. For example, we used an integer array view,
    which is obtained by calling \verb+Int32Array+. Using this, we
    wrote a simple memory allocation function called \texttt{memAlloc}. This function takes the given
    size and allocates size bytes from the last allocation. There is currently no way to
    free this memory, until the program exits and the allocated heap is collected by the JS 
    garbage collector. 

    Memory is an interesting concept in Asm.js. The language borrows from techniques pioneered by 
    Emscripten and Mandril, by forcing the programmer to guarantee that any offset into the array
    is `aligned'. This is done by forcing the user to make all array accesses as follows:
\begin{verbatim}
  H32[loc >> 2]
\end{verbatim}
    This, however, is hardly mentioned in the spec, and woefully undocumented. We only discovered
    this necessity late into the final days of our project, through discussing the issue on the
    asm.js irc. A fix, which we recently thought of, which allows you to meet the dubious Asm.js 
    spec and still simulate array accesses correctly is as follows: 
\begin{verbatim}
  H32[(loc << 2) >> 2]
\end{verbatim}
    As after the first left-shift by 2, we transform loc, an \texttt{int} into \texttt{intish}, 
    we ensure that the array shift by two now both satisfies the Asm.js validator in Firefox,
    but also maintains the heap-view properties we desire when using pointers. As we wanted
    to ensure correctness in any engine, we decided to leave this slightly dubious hack out
    of the submitted project, although we guarantee that applying this tiny bug-fix to all
    memory accessing functions in the library we provide (there's about 10) will cause Firefox
    to validate and Ahead-Of-Time compile the Asm.js code we generate. 

  \subsection{Pointers}
    Pointers are merely integers that represent the base index at which the
    the memory is stored in the H32 array.

    Since the NULL address exists in the H32 heap array, it is necessary for
    us to check if pointer dereferences are NULL, and setting the correct
    exception flag. We decided to create a JS function that would handle all
    pointer dereferences that handles this. 

    Furthermore - when compiling pointers - some situations call for actually
    dereferencing the heap, and other situations call for merely verifying that 
    the offset into the heap is non-zero. An example of the latter case is when
    the object we dereference is a struct, and the next operation we wish to do
    on this struct is to load one of its fields. Because of this need, we provide
    two functions that operate on pointers - \texttt{pointerLoad} and \texttt{pointerDeref}. 
    The former actually fetches the value stored in the heap, and the latter
    merely verifies that the offset into the heap is non-zero, and returns the offset. 

  \subsection{Arrays}
    Arrays are allocated with the \texttt{memArrAlloc} JavaScript function. This
    works like \texttt{memAlloc} but checks that the number of elements is
    positive, setting the memory exception global variable if not. The number of elements
    is then stored in the base of the array. Once an array has been allocated, we increment
    a global heap-counter variable, that increases by the size of the allocated array.

    All array subscripts check that the index is between 0 and the size of the
    array that is stored at the base of the array. This is handled by the
    \texttt{arrAccess} and \texttt{arrShift} functions, which accesses the array
    and calculates the address of the array respectively.

  \subsection{Structs}
    This is identical to how structures are implemented in previous labs. Once a struct
    is allocated, we return a pointer into the heap where the struct is stored. As with
    arrays, we increment a global counter heap-counter variable that increases by the 
    size of the allocated struct. 

  \subsection{External Library}
    The external library is implemented in JavaScript, and imported in as foreign
    imported functions. The foreign imports used by default are the implementation 
    of the function calls defined in \texttt{15411.h0}, the header file for the 
    standard library introduced in lab3. 

  \subsection{Exceptions}
    Unfortunately, Asm.js does not support the \texttt{try} / \texttt{catch}
    statements in JavaScript.
    Instead, exceptions are flagged by setting either the \texttt{g\_memex} global variable
    for memory exceptions, \texttt{g\_oomex} global variable for out of memory
    exceptions, \texttt{g\_numex} for number exceptions, and \texttt{g\_assex} for
    assertion exceptions.

%------------------------------------------------

\section{Testing Methodology}

\subsection{Testing Directory}
Tests are collected into the \texttt{js\_tests} directory. They are organized
by the lab number they originate from, and whether they are return tests or not. 
We provide further segregation for labs 3 and 4 for special types of tests that
we will now elaborate. 

\subsection{Test Selection}
Since we are merely re-targeting the L4 language to Asm.js, we used the L4 test
files from Lab 4 to test our compiler. As was in Lab 4, these tests provide a
wide array of various cases that ensures that our compiler is correct and
sufficiently fast. We also ensure backwards compatibility by providing return
tests from L1, L2 and L3. 

\begin{enumerate}

  \item Return Tests

  We provide return tests for L1, L2, L3 and L4, in the directories: 
  \texttt{js\_tests/L[1-4]} for each lab respectively. One may run these 
  tests (the L4 tests, for example) using the following command to \texttt{tester.py}
\begin{verbatim}
  ./tester.py -d l4
\end{verbatim}

  \item Error and Exception Tests
  We provide error and exception tests for L1, L2, L3 and L4 in the direcotries:
  \texttt{js\_tests/L[1-4]err} for each lab respectively. One may run these
  tests (the L4 error tests, for example) using the following command to \texttt{tester.py}
\begin{verbatim}
  ./tester.py -e -d l4
\end{verbatim}
  The \texttt{-e} flag specifies that these tests should timeout after a default
  timeout of 10 seconds - when compiled without the -e flag, any tests that do not
  halt will run indefinitely. 

  \item Heap Tests
  We segregate L4 tests which require a large amount of heap into the 
  \texttt{js\_tests/l4heap} directory. In order for these tests to compute
  the correct result, with no exceptions, they must be instantiated with a fairly
  large heap. This can be done as follows: Once the test is compiled, using either
  a command to the compiler, or a simple \texttt{./tester.py -f filename} command,
  one can open up the \texttt{filename.js} file, and increase the size of the 
  \texttt{ArrayBuffer} that is initially passed into the asm module. The two lines
  which need to be modified are in particular: 
\begin{verbatim}
var c0arr = new Int32Array(5000000)
c0arr[0] = 5000000;
\end{verbatim}
  We reccommend setting the value to 500 Mb and re-running the test using 
  \texttt{js filename.js}. 
\end{enumerate}

We decided to skip the following types of tests:

\begin{enumerate}

  \item Floating Point Library Tests

    These tests strenuously check that external library functions are correct.
    We know from our other tests that external library functions are called
    correctly, but our library functions themselves are not 100\% correct.
    The library functions themselves are uninteresting from the compiler's
    aspect. We have placed these tests into \texttt{js\_tests/fpt}

  \item Deep Recursive Calls

    A few tests recurse too deep for our build of SpiderMonkey to handle. These
    tests fail because of SpiderMonkey and not from our code. The engine can be 
    modified in a fairly straightforward manner to support a large stack size (
    and therefore support these tests - however, we wanted to operate under the 
    assumption that our engine and browser are all stock). We have placed
    these tests into \texttt{js\_tests/recur}. 

\end{enumerate}

\subsection{Testing Framework}

We needed to write a new testing framework, \texttt{tester.py}, which takes in either a
file or a directory of files to test. This framework makes a call to \texttt{js},
which should be the preferred JavaScript engine to run the compiled Asm.js code.
We predominantly used the SpiderMonkey engine during the development of our
compiler.

For each L4 file, \texttt{tester.py} will compile the file by running \texttt{bin/l4c} 
using the \texttt{asmjs} directive. A compilation error will fail out the test. There 
should now be a compiled \texttt{.js} file of the test.

This \texttt{.js} file is then run using the \texttt{js} command. The return
result is then checked with the expected return result, giving us a test pass
or fail.

There is no timeout limit. Tests will keep running until completion, but can be
skipped with \texttt{CTRL+C}.

\subsection{Validating Correctness}

Our testing framework does a good job of validating that the compiled JavaScript
files return the correct result. It parses the top of the test file to see whether
the test is a \texttt{return} test, a \texttt{error} test or an \texttt{exception} 
test. For \texttt{return} tests, we ensure that the answer returned from running 
the produced \texttt{.js} file is the same as the answer at the top of the C0 file.
For error tests, we ensure that the compiler produces a compilation error. For 
exception tests, we ensure that either 1) the dynamic execution of the code times
out, or 2) the code executes correctly, and produces one of the exception flags
(numerical exception, memory exception, or assertion exception). 


\section{Conforming to Asm.js}
The spidermonkey engine we used does not check if our produced JavaScript files
conform to the Asm.js specifications. To do this, we had to manually paste the
outputted code into \emph{http://turtlescript.github.cscott.net/asmjs.html}, one of a 
handful of Asm.js validators. 

We also note that a node.js module is available for validating correctness, 
but it is buggy and does not meet the spec currently implemented in Firefox 25.0.

Regarding Asm.js correctness, we want to clarify that the compiler we submitted
meets the Asm.js standard everywhere except in two areas. These areas are:

\begin{enumerate}
  \item Heap Access: 
    We mentioned earlier that heap accesses in Asm.js must be `aligned'. This
    appears to derive from some techniques used in the Emscripten and Mandril
    compilers. The idea is that all heap accesses must be suffixed by a \texttt{>> 2}.
    We realized that one way to meet this requirement for valid Asm.js while still
    maintaining the overall correctness of our comppiler was to emit heap accesses
    of the form \texttt{H32[loc << 2] >> 2}. Testing our compiler with the memory
    accessing functions using this version of heap access produced no errors, but
    we decided to leave it out of the final submitted version, and only mention it
    as it felt like a way to hack around the language spec. 

  \item \texttt{\&\&, ||}
  Asm.js does not support the \texttt{\&\&} and \texttt{||} operations. Our compiler
  therefore emits \texttt{\&} and \texttt{|} - as booleans are just ints, this 
  works out fine, and correctness is maintained. However, in certain situations, such
  as library functions that are called many times, we continue to use \texttt{\&\&} and
  \texttt{||} as for backends that do not support Asm.js, replacing \texttt{\&} with 
  \texttt{\&\&} is actually quite costly, and makes the generated code significantly slower. 
  This is a quick fix, and can easily be fixed in the statically linked javascript to 
  make the code entirely Asm.js compliant. 
\end{enumerate}
We want to emphasize that aside from these two caveats, our code is fully Asm.js 
compliant. Our reason for not submitting these two tiny fixes with our compiler is
that the former feels like a hacky way to get around a poorly explained feature of
Asm.js, and that the latter causes significant performance regressions on backends
that do not support Asm.js. 


%------------------------------------------------

\section{Analysis}

\subsection{Evaluation of Compiler}

Overall, we were pretty successful with the compiler. Aside from the aforementioned
tests that we skipped, we pass pretty much every test. This hits our number one
goal of the project.

There are a few issues, however. Many tests take a significant amount of time to
complete. It is expected that running JS is slower than running x86 Assembly.
That said, it's unclear whether the bottleneck lies in our compiled JS or the
JavaScript engine.

\subsection{Future Improvements}

There are a few additions to the L4 language that would not be that difficult
to do, but were beyond the scope of this lab:

\begin{enumerate}
  \item Break / Continue in loops. Asm.js supports these statements natively.
    This would be simple in the back-end, but may take significant rework in the
    front-end code. This is a fairly important construct that would be very useful
    to have in a future iteration.
  \item Switch Statements. As above, this is supported in Asm.js, and may take
    reworking on the front-end side. This is not as important as the above break
    / continue, but could make for more readable / efficient code.
  \item Native Float Type. Asm.js supports doubles.
  \item More Library Functions. Asm.js supports various Math standard library
    calls.
\end{enumerate}

Other potential improvements not in Asm.js include:

\begin{enumerate}
  \item Garbage Collection. No memory is recovered at the moment, which means
    we may run out of memory sooner than we should.
  \item Optimizations. As mentioned in our evaluation of our compiler, many tests
    run extremely slowly. A potential improvement is to look into various
    that we can perform to speed up our compiled JavaScript.
\end{enumerate}

%----------------------------------------------------------------------------------------

\end{multicols}

\end{document}
