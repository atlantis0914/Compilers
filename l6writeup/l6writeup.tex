\documentclass[twoside]{article}

\usepackage{lipsum} % Package to generate dummy text throughout this template

\usepackage[sc]{mathpazo} % Use the Palatino font
\usepackage[T1]{fontenc} % Use 8-bit encoding that has 256 glyphs
\linespread{1.05} % Line spacing - Palatino needs more space between lines
\usepackage{microtype} % Slightly tweak font spacing for aesthetics

\usepackage[hmarginratio=1:1,top=32mm,columnsep=20pt]{geometry} % Document margins
\usepackage{multicol} % Used for the two-column layout of the document
\usepackage[hang, small,labelfont=bf,up,textfont=it,up]{caption} % Custom captions under/above floats in tables or figures
\usepackage{booktabs} % Horizontal rules in tables
\usepackage{float} % Required for tables and figures in the multi-column environment - they need to be placed in specific locations with the [H] (e.g. \begin{table}[H])
\usepackage{hyperref} % For hyperlinks in the PDF

\usepackage{lettrine} % The lettrine is the first enlarged letter at the beginning of the text
\usepackage{paralist} % Used for the compactitem environment which makes bullet points with less space between them

\usepackage{abstract} % Allows abstract customization
\renewcommand{\abstractnamefont}{\normalfont\bfseries} % Set the "Abstract" text to bold
\renewcommand{\abstracttextfont}{\normalfont\small\itshape} % Set the abstract itself to small italic text

\usepackage{titlesec} % Allows customization of titles
\renewcommand\thesection{\Roman{section}} % Roman numerals for the sections
\renewcommand\thesubsection{\Roman{subsection}} % Roman numerals for subsections
\titleformat{\section}[block]{\large\scshape\centering}{\thesection.}{1em}{} % Change the look of the section titles
\titleformat{\subsection}[block]{\large}{\thesubsection.}{1em}{} % Change the look of the section titles

\usepackage{fancyhdr} % Headers and footers
\pagestyle{fancy} % All pages have headers and footers
\fancyhead{} % Blank out the default header
\fancyfoot{} % Blank out the default footer
\fancyfoot[RO,LE]{\thepage} % Custom footer text

%----------------------------------------------------------------------------------------
%	TITLE SECTION
%----------------------------------------------------------------------------------------

\title{\vspace{-15mm}\fontsize{20pt}{10pt}\selectfont\textbf{Lab 6 Compiler}} % Article title

\author{
\large
\textsc{Laxman Dhulipala, Peter Xiao}\\[2mm] % Your name
\normalsize Carnegie Mellon University \\ % Your institution
\normalsize \href{mailto:ldhulipa@andrew.cmu.edu}{ldhulipa@andrew.cmu.edu}, \href{mailto:phx@andrew.cmu.edu}{phx@andrew.cmu.edu} % Your email address
\vspace{-5mm}
}
\date{}

%----------------------------------------------------------------------------------------

\begin{document}

\maketitle % Insert title

\thispagestyle{fancy} % All pages have headers and footers

%----------------------------------------------------------------------------------------
%	ABSTRACT
%----------------------------------------------------------------------------------------

\begin{abstract}

\noindent We evaluate an experimental compiler that translates C0 code into Asm.js, a strict
and highly optimizable subset of javascript. We describe the results both holistically and emperically,
profiling performance using the Spidermonkey and V8 javascript engines. Importantly, in the
translation, we preserve all dynamic semantics guaranteed by C0 and guarantee deterministic execution.

\end{abstract}

%----------------------------------------------------------------------------------------
%	ARTICLE CONTENTS
%----------------------------------------------------------------------------------------

\begin{multicols}{2} % Two-column layout throughout the main article text

\section{Introduction}

%\lettrine[nindent=0em,lines=3]{L} orem ipsum dolor sit amet, consectetur adipiscing elit.
%\lipsum[2-3] % Dummy text

Asm.js is a recent effort by Mozilla to develop a highly optimizable and performant
subset of javascript. The motivation behind the effort stems from a series of projects which
compile C and C++ to javascript, most notably the Emscripten compiler. With the availability of such
compilers, the only missing features for browsers was an engine which could run the
translated native code at near-native speeds. Asm.js is Mozilla's solution to this problem,
and provides a directive triggered mechanism by which the browser can defer control to a
self-contained, non-garbage collected module that runs translated C and C++ at near-native speeds. \\
\\
As we will shortly see, Asm.js is not a language that is `fun' or `easy' for humans to write in -
its use is primarily as a compiler target. The current mechanism for compiling C, and C++ to
javascript is as follows:

\begin{compactitem}
  \item C and C++ is compiled to the LLVM IR
  \item The LLVM IR is handed to Emscripten, a LLVM to Javascript compiler
  \item Emscripten compiles the IR to a portable asm.js module
\end{compactitem}


%------------------------------------------------

\section{Project Specification}

At a high level, this project will retarget our L4 compiler to Asm.js. In other
words, the following features C0 will be supported:

    Memory is supported as in L4 with no garbage collection.

    All of the external libraries in C0 are reimplemented in JS.

As always, correctness is paramount. Our number one goal is to create a correctly
functioning compiler.

%------------------------------------------------

\section{Implementation}

  \subsection{Loops and Conditionals}
  \subsection{Functions}
  \subsection{Memory}
    Asm.js provides a large binary heap, in the form of an array called H32, for which we
    wrote a simple memory allocation function for. This function takes the given
    size and allocates size bytes from the last allocation. There is no way to
    free this memory.
  \subsection{Pointers}
    Pointers are merely integers that represent the base index at which the
    the memory is stored in the H32 array.

    Since the NULL address exists in the H32 heap array, it is necessary for
    us to check if pointer dereferences are NULL, and setting the correct
    exception flag. We decided to create a JS function that would handle all
    pointer dereferences that handles this.
  \subsection{Arrays}
  \subsection{Structs}
  \subsection{External Library}
  \subsection{Exceptions}

%------------------------------------------------

\section{Testing Methodology}

\subsection{Test Selection}
Since we are merely retargeting the L4 language to Asm.js, we used the L4 test
files from Lab 4 to test our compiler. As was in Lab 4, these tests provide a
wide array of various cases that ensures that our compiler is correct and
sufficiently fast.

We decided to skip a the following types of tests:

\begin{enumerate}
  \item Floating Point Library Tests

    These tests strenuously check that external library functions are correct.
    We know from our other tests that external library functions are called
    correctly, but our library functions themselves are not 100\% correct.
    The library functions themselves are uninteresting from the compiler's
    aspect.

  \item Deep Recursive Calls

    A few tests recurse too deep for our build of SpiderMonkey to handle. These
    tests fail because of SpiderMonkey and not from our code.
\end{enumerate}

\subsection{Testing Framework}

We needed to write a new testing framework, tester.py, which takes in either a
file or a directory of files to test.

For each L4 file, tester.py will compile the file using the \texttt{--asmjs}
directive.

%------------------------------------------------

\section{Analysis}



%----------------------------------------------------------------------------------------

\end{multicols}

\end{document}
