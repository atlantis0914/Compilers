\documentclass[twoside]{article}

\usepackage{lipsum} % Package to generate dummy text throughout this template

\usepackage[sc]{mathpazo} % Use the Palatino font
\usepackage[T1]{fontenc} % Use 8-bit encoding that has 256 glyphs
\linespread{1.05} % Line spacing - Palatino needs more space between lines
\usepackage{microtype} % Slightly tweak font spacing for aesthetics

\usepackage[hmarginratio=1:1,top=32mm,columnsep=20pt]{geometry} % Document margins
\usepackage{multicol} % Used for the two-column layout of the document
\usepackage[hang, small,labelfont=bf,up,textfont=it,up]{caption} % Custom captions under/above floats in tables or figures
\usepackage{booktabs} % Horizontal rules in tables
\usepackage{float} % Required for tables and figures in the multi-column environment - they need to be placed in specific locations with the [H] (e.g. \begin{table}[H])
\usepackage{hyperref} % For hyperlinks in the PDF

\usepackage{lettrine} % The lettrine is the first enlarged letter at the beginning of the text
\usepackage{paralist} % Used for the compactitem environment which makes bullet points with less space between them

\usepackage{abstract} % Allows abstract customization
\renewcommand{\abstractnamefont}{\normalfont\bfseries} % Set the "Abstract" text to bold
\renewcommand{\abstracttextfont}{\normalfont\small\itshape} % Set the abstract itself to small italic text

\usepackage{titlesec} % Allows customization of titles
\renewcommand\thesection{\Roman{section}} % Roman numerals for the sections
\renewcommand\thesubsection{\Roman{subsection}} % Roman numerals for subsections
\titleformat{\section}[block]{\large\scshape\centering}{\thesection.}{1em}{} % Change the look of the section titles
\titleformat{\subsection}[block]{\large}{\thesubsection.}{1em}{} % Change the look of the section titles

\usepackage{fancyhdr} % Headers and footers
\pagestyle{fancy} % All pages have headers and footers
\fancyhead{} % Blank out the default header
\fancyfoot{} % Blank out the default footer
\fancyfoot[RO,LE]{\thepage} % Custom footer text

%----------------------------------------------------------------------------------------
%	TITLE SECTION
%----------------------------------------------------------------------------------------

\title{\vspace{-15mm}\fontsize{20pt}{10pt}\selectfont\textbf{Lab 6 Compiler}} % Article title

\author{
\large
\textsc{Laxman Dhulipala, Peter Xiao}\\[2mm] % Your name
\normalsize Carnegie Mellon University \\ % Your institution
\normalsize \href{mailto:ldhulipa@andrew.cmu.edu}{ldhulipa@andrew.cmu.edu}, \href{mailto:phx@andrew.cmu.edu}{phx@andrew.cmu.edu} % Your email address
\vspace{-5mm}
}
\date{}

%----------------------------------------------------------------------------------------

\begin{document}

\maketitle % Insert title

\thispagestyle{fancy} % All pages have headers and footers

%----------------------------------------------------------------------------------------
%	ABSTRACT
%----------------------------------------------------------------------------------------

\begin{abstract}

\noindent We evaluate an experimental compiler that translates C0 code into Asm.js, a strict 
and highly optimizable subset of javascript. We describe the results both holistically and emperically, 
profiling performance using the Spidermonkey and V8 javascript engines. Importantly, in the
translation, we preserve all dynamic semantics guaranteed by C0 and guarantee deterministic execution. 

\end{abstract}

%----------------------------------------------------------------------------------------
%	ARTICLE CONTENTS
%----------------------------------------------------------------------------------------

\begin{multicols}{2} % Two-column layout throughout the main article text

\section{Introduction}

%\lettrine[nindent=0em,lines=3]{L} orem ipsum dolor sit amet, consectetur adipiscing elit.
%\lipsum[2-3] % Dummy text

Asm.js is a recent effort by Mozilla to develop a highly optimizable and performant 
subset of javascript. The motivation behind the effort stems from a series of projects which
compile C and C++ to javascript, most notably the Emscripten compiler. With the availability of such
compilers, the only missing features for browsers was an engine which could run the 
translated native code at near-native speeds. Asm.js is Mozilla's solution to this problem,
and provides a directive triggered mechanism by which the browser can defer control to a 
self-contained, non-garbage collected module that runs translated C and C++ at near-native speeds. \\
\\
As we will shortly see, Asm.js is not a language that is `fun' or `easy' for humans to write in - 
its use is primarily as a compiler target. The current mechanism for compiling C, and C++ to
javascript is as follows: 

\begin{compactitem}
  \item C and C++ is compiled to the LLVM IR
  \item The LLVM IR is handed to Emscripten, a LLVM to Javascript compiler
  \item Emscripten compiles the IR to a portable asm.js module
\end{compactitem}


%------------------------------------------------

\section{Project Specification}

\begin{enumerate}
  \item External Library
  \item Loops and Conditionals
  \item Functions
  \item Pointers
  \item Arrays
  \item Structs
\end{enumerate}

%------------------------------------------------

\section{Implementation}

\begin{enumerate}
  \item External Library
  \item Loops and Conditionals
  \item Functions
  \item Pointers
  \item Arrays
  \item Structs
\end{enumerate}

%------------------------------------------------

\section{Testing Methodology}

%------------------------------------------------

\section{Analysis}

%----------------------------------------------------------------------------------------
%	REFERENCE LIST
%----------------------------------------------------------------------------------------

\begin{thebibliography}{99} % Bibliography - this is intentionally simple in this template

\bibitem[Figueredo and Wolf, 2009]{Figueredo:2009dg}
Figueredo, A.~J. and Wolf, P. S.~A. (2009).
\newblock Assortative pairing and life history strategy - a cross-cultural
  study.
\newblock {\em Human Nature}, 20:317--330.
 
\end{thebibliography}

%----------------------------------------------------------------------------------------

\end{multicols}

\end{document}
