\documentclass[twoside]{article}

\usepackage{lipsum} % Package to generate dummy text throughout this template

\usepackage[sc]{mathpazo} % Use the Palatino font
\usepackage[T1]{fontenc} % Use 8-bit encoding that has 256 glyphs
\linespread{1.05} % Line spacing - Palatino needs more space between lines
\usepackage{microtype} % Slightly tweak font spacing for aesthetics

\usepackage[hmarginratio=1:1,top=32mm,columnsep=20pt]{geometry} % Document margins
\usepackage{multicol} % Used for the two-column layout of the document
\usepackage[hang, small,labelfont=bf,up,textfont=it,up]{caption} % Custom captions under/above floats in tables or figures
\usepackage{booktabs} % Horizontal rules in tables
\usepackage{float} % Required for tables and figures in the multi-column environment - they need to be placed in specific locations with the [H] (e.g. \begin{table}[H])
\usepackage{hyperref} % For hyperlinks in the PDF

\usepackage{lettrine} % The lettrine is the first enlarged letter at the beginning of the text
\usepackage{paralist} % Used for the compactitem environment which makes bullet points with less space between them

\usepackage{abstract} % Allows abstract customization
\renewcommand{\abstractnamefont}{\normalfont\bfseries} % Set the "Abstract" text to bold
\renewcommand{\abstracttextfont}{\normalfont\small\itshape} % Set the abstract itself to small italic text

\usepackage{titlesec} % Allows customization of titles
\renewcommand\thesection{\Roman{section}} % Roman numerals for the sections
\renewcommand\thesubsection{\Roman{subsection}} % Roman numerals for subsections
\titleformat{\section}[block]{\large\scshape\centering}{\thesection.}{1em}{} % Change the look of the section titles
\titleformat{\subsection}[block]{\large}{\thesubsection.}{1em}{} % Change the look of the section titles

\usepackage{fancyhdr} % Headers and footers
\pagestyle{fancy} % All pages have headers and footers
\fancyhead{} % Blank out the default header
\fancyfoot{} % Blank out the default footer
\fancyfoot[RO,LE]{\thepage} % Custom footer text

%----------------------------------------------------------------------------------------
%	TITLE SECTION
%----------------------------------------------------------------------------------------

\title{\vspace{-15mm}\fontsize{20pt}{10pt}\selectfont\textbf{Lab 6 Compiler}} % Article title

\author{
\large
\textsc{Laxman Dhulipala, Peter Xiao}\\[2mm] % Your name
\normalsize Carnegie Mellon University \\ % Your institution
\normalsize \href{mailto:ldhulipa@andrew.cmu.edu}{ldhulipa@andrew.cmu.edu}, \href{mailto:phx@andrew.cmu.edu}{phx@andrew.cmu.edu} % Your email address
\vspace{-5mm}
}
\date{}

%----------------------------------------------------------------------------------------

\begin{document}

\maketitle % Insert title

\thispagestyle{fancy} % All pages have headers and footers

%----------------------------------------------------------------------------------------
%	ABSTRACT
%----------------------------------------------------------------------------------------

\begin{abstract}

\noindent We evaluate an experimental compiler that translates C0 code into Asm.js, a strict
and highly optimizable subset of Javascript. We describe the results both holistically and empirically,
profiling performance using the Spidermonkey and V8 JavaScript engines. Importantly, in the
translation, we preserve all dynamic semantics guaranteed by C0 and guarantee deterministic execution.

\end{abstract}

%----------------------------------------------------------------------------------------
%	ARTICLE CONTENTS
%----------------------------------------------------------------------------------------

\begin{multicols}{2} % Two-column layout throughout the main article text

\section{Introduction}

%\lettrine[nindent=0em,lines=3]{L} orem ipsum dolor sit amet, consectetur adipiscing elit.
%\lipsum[2-3] % Dummy text

Asm.js is a recent effort by Mozilla to develop a highly optimizable and performant
subset of JavaScript. The motivation behind the effort stems from a series of projects which
compile C and C++ to JavaScript, most notably the Emscripten compiler. With the availability of such
compilers, the only missing features for browsers was an engine which could run the
translated native code at near-native speeds. Asm.js is Mozilla's solution to this problem,
and provides a directive triggered mechanism by which the browser can defer control to a
self-contained, non-garbage collected module that runs translated C and C++ at near-native speeds. \\
\\
As we will shortly see, Asm.js is not a language that is `fun' or `easy' for humans to write in -
its use is primarily as a compiler target. The current mechanism for compiling C, and C++ to
JavaScript is as follows:

\begin{compactitem}
  \item C and C++ is compiled to the LLVM IR
  \item The LLVM IR is handed to Emscripten, a LLVM to JavaScript compiler
  \item Emscripten compiles the IR to a portable asm.js module
\end{compactitem}


%------------------------------------------------

\section{Project Specification}

At a high level, this project will re-target our L4 compiler to JavaScript that
conforms to Asm.js specifications. In other
words, all of the following features C0 will be supported: loops, conditionals,
functions, memory, pointers, arrays, structs, external library functions, and
exceptions.

  Memory is supported as in L4 with no garbage collection. Exceptions are
  still thrown as they would in previous labs, but execution of the program
  will not halt.

  All of the external libraries in C0 are reimplemented in JS.

As always, correctness is paramount. Our number one goal is to create a correctly
functioning compiler. How we assess that our compiler is correct is outlined in
section IV.

%------------------------------------------------

\section{Implementation}

  \subsection{Integers}
    All variables in asm.js must be explicitly annotated with type information.
    This is so that their type will be statically enforced.

    Integers are annotated with an \texttt{| 0} appended to the end of any
    assignment or return.
  \subsection{Loops and Conditionals}
    These are very similar to their C0 counterparts.
  \subsection{Functions}
    Functions in Asm.js are very similar to their C0 counterparts, but with a
    slightly different syntax. Function calls require explicit coercions.
  \subsection{Memory}
    Asm.js provides a large binary heap, in the form of an array called H32, for which we
    wrote a simple memory allocation function for \texttt{memAlloc}. This function takes the given
    size and allocates size bytes from the last allocation. There is no way to
    free this memory.
  \subsection{Pointers}
    Pointers are merely integers that represent the base index at which the
    the memory is stored in the H32 array.

    Since the NULL address exists in the H32 heap array, it is necessary for
    us to check if pointer dereferences are NULL, and setting the correct
    exception flag. We decided to create a JS function that would handle all
    pointer dereferences that handles this.
  \subsection{Arrays}
    Arrays are allocated with the \texttt{memArrAlloc} JavaScript function. This
    works like \texttt{memAlloc} but checks that the number of elements is
    positive, setting the exception global variable if not. The number of elements
    is then stored in the base of the array.

    All array subscripts check that the index is between 0 and the size of the
    array that is stored at the base of the array. This is handled by the
    \texttt{arrAccess} and \texttt{arrShift} functions, which accesses the array
    and calculates the address of the array respectively.
  \subsection{Structs}
    Similar to how structures are implemented in previous labs.
  \subsection{External Library}
    The external library is implemented in JavaScript, and imported in as foreign
    imported functions.
  \subsection{Exceptions}
    Unfortunately, Asm.js does not support the \texttt{try} / \texttt{catch}
    statements in JavaScript.
    Instead, exceptions are flagged by setting either the \texttt{g\_memex} global variable
    for memory exceptions, \texttt{g\_oomex} global variable for out of memory
    exceptions, \texttt{g\_numex} for number exceptions, and \texttt{g\_assex} for
    assertion exceptions.

%------------------------------------------------

\section{Testing Methodology}

\subsection{Test Selection}
Since we are merely re-targeting the L4 language to Asm.js, we used the L4 test
files from Lab 4 to test our compiler. As was in Lab 4, these tests provide a
wide array of various cases that ensures that our compiler is correct and
sufficiently fast.

We decided to skip the following types of tests:

\begin{enumerate}
  \item Error Tests

    The front-end of our compiler is exactly the same as in Lab 5, so all error
    cases should be the same. This will simplify our testing framework.

  \item Floating Point Library Tests

    These tests strenuously check that external library functions are correct.
    We know from our other tests that external library functions are called
    correctly, but our library functions themselves are not 100\% correct.
    The library functions themselves are uninteresting from the compiler's
    aspect.

  \item Deep Recursive Calls

    A few tests recurse too deep for our build of SpiderMonkey to handle. These
    tests fail because of SpiderMonkey and not from our code.
\end{enumerate}

\subsection{Testing Framework}

We needed to write a new testing framework, \texttt{tester.py}, which takes in either a
file or a directory of files to test. This framework makes a call to \texttt{js},
which should be the preferred JavaScript engine to run the compiled Asm.js code.
We predominantly used the SpiderMonkey engine during the development of our
compiler.

For each L4 file, \texttt{tester.py} will compile the file using the \texttt{asmjs}
directive. A compilation error will fail out the test. There should now be a
compiled \texttt{.js} file of the test.

This \texttt{.js} file is then run using the \texttt{js} command. The return
result is then checked with the expected return result, giving us a test pass
or fail.

There is no timeout limit. Tests will keep running until completion, but can be
skipped with \texttt{CTRL+C}.

\subsection{Validating Correctness}

Our testing framework does a good job of validating that the compiled JavaScript
files return the correct result, but it does not check if these JavaScript files
conform to the Asm.js specifications. To do this, we had to manually paste the
outputted code into http://turtlescript.github.cscott.net/asmjs.html.

%------------------------------------------------

\section{Analysis}

\subsection{Evaluation of Compiler}

Overall, we were pretty successful with the compiler. Aside from the aforementioned
tests that we skipped, we pass pretty much every test. This hits our number one
goal of the project.

There are a few issues, however. Many tests take a significant amount of time to
complete. It is expected that running JS is slower than running x86 Assembly.
That said, it's unclear whether the bottleneck lies in our compiled JS or the
JavaScript engine.

\subsection{Future Improvements}

There are a few additions to the L4 language that would not be that difficult
to do, but were beyond the scope of this lab:

\begin{enumerate}
  \item Break / Continue in loops. Asm.js supports these statements natively.
    This would be simple in the back-end, but may take significant rework in the
    front-end code. This is a fairly important construct that would be very useful
    to have in a future iteration.
  \item Switch Statements. As above, this is supported in Asm.js, and may take
    reworking on the front-end side. This is not as important as the above break
    / continue, but could make for more readable / efficient code.
  \item Native Float Type. Asm.js supports doubles.
  \item More Library Functions. Asm.js supports various Math standard library
    calls.
\end{enumerate}

Other potential improvements not in Asm.js include:

\begin{enumerate}
  \item Garbage Collection. No memory is recovered at the moment, which means
    we may run out of memory sooner than we should.
  \item Optimizations. As mentioned in our evaluation of our compiler, many tests
    run extremely slowly. A potential improvement is to look into various
    that we can perform to speed up our compiled JavaScript.
\end{enumerate}

%----------------------------------------------------------------------------------------

\end{multicols}

\end{document}
