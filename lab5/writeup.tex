% You should title the file with a .tex extension (hw1.tex, for example)
\documentclass[11pt]{article}

\usepackage{amsmath}
\usepackage{amssymb}
\usepackage{fancyhdr}
\usepackage{booktabs}


\oddsidemargin0cm
\topmargin-1.5cm     %I recommend adding these three lines to increase the
\textwidth16.5cm   %amount of usable space on the page (and save trees)
\textheight23.5cm

\newcommand{\question}[2] {\vspace{.25in} \hrule\vspace{0.5em}
\noindent{\bf #1: #2} \vspace{0.5em}
\hrule \vspace{.10in}}
\renewcommand{\part}[1] {\vspace{.10in} {\bf (#1)}}

\newcommand{\myname}{Laxman Dhulipala, Peter Xiao}
\newcommand{\myandrew}{ldhulipa, phx}
\newcommand{\myhwnum}{5}

\setlength{\parindent}{0pt}
\setlength{\parskip}{5pt plus 1pt}

\pagestyle{fancyplain}


\begin{document}

\medskip                        % Skip a "medium" amount of space
                                % (latex determines what medium is)
                                % Also try: \bigskip, \littleskip

\thispagestyle{plain}
\begin{center}                  % Center the following lines
{\Large Lab 5 Writeup} \\
 \vspace{2 mm}
\myname \\
\myandrew
\end{center}

For this lab we implemented several optimizations that range from function inlining
to register coalescing. Specifically, we picked the following four optimizations to focus
on : 

\begin{itemize}
  \item[1] Dead Code Elimination
  \item[2] Function Inlining
  \item[3] Constant Folding and Propagation
  \item[4] Register Coalescing
\end{itemize}

In the following sections, we will describe in detail the implementation and evaluation of 
each optimization. 

\section{Dead Code Elimination}

\section{Function Inlining}

\section{Constant Folding and Propagation}

\section{Register Coalescing}

\end{document}
